\documentclass[12pt]{article}
\usepackage{geometry}
\usepackage{amsmath}

%opening
\title{}
\author{}

\begin{document}

Indeed there can be applied an existing $k$-d tree method for density estimation and it will speed up the computation of the objective function: $\sum_i{p(c_i|\mathbf{x}_i)}$. What I am arguing is that for the gradient computation is not that simple to apply the same ideas.

From what I understand, for a query point $\mathbf{x}_q$ and a group of points $G$, there are basically two types of approximations that can be done:
\begin{itemize}
  \item If the points $\mathbf{x}_i$ in $G$ are very close together we can approximate each individual contribution $k_A(\mathbf{x}_q, \mathbf{x}_i)$ with some sort of an average contribution, let's say $k_A(\mathbf{x}_q, \mathbf{\mu}_G)$.
  \item If the points $\mathbf{x}_i$ in $G$ are very far from the query point $\mathbf{x}_q$, we can ignore their contribution: $k_A(\mathbf{x}_i,\mathbf{x}_q) = 0, \forall \mathbf{x}_i \in G$.
\end{itemize}

I think that the first type of approximation won't give any improvements for the gradient computation.

The derivative consists of sums of the following form: $S_i = \sum_i k_A(\mathbf{x}_q,\mathbf{x}_i) \mathbf{x}_{qi} \mathbf{x}_{qi}^\mathrm{T}$, where $\mathbf{x}_{qi}$ are $D$-dimensional column vectors. If I try to use the first type of pruning $S_i \approx k_A(\mathbf{x}_q, \mu_G) \sum_i \mathbf{x}_{qi} \mathbf{x}_{qi}^\mathrm{T}$ then, because I cannot really cache those matrices, I will still have to compute their values for each $q$ and $i$ and for each iteration; the complexity will remain $O(ND^2)$. An idea might be to also approximate $\mathbf{x}_{qi}$ with an average version of the group $\mathbf{x}_{qG}$, but this is not really correct, because the points $\mathbf{x}_i$ form a node of the $k$-d tree in the projected space, whereas $\mathbf{x}_{qi}$ refer to the points in the original space.

On the other hand, the second type of approximation, that excludes some of the points, will give some speed-ups for the gradient because we don't have to compute all the statistics $\mathbf{x}_{qi} \mathbf{x}_{qi}^\mathrm{T}$. Also lower and upper bounds can be introduced to get a sense of the maximum error that can result. Moreover, we can accumulate weights in a best-first search and stop when the sum doesn't change significantly.

For convenience, here are the equations of the objective function and the gradient:
\begin{align}
 f(A) &= \frac{1}{N}\sum_{i=1}^Np_i=\frac{1}{N}\sum_{i=1}^N\sum_{j\in\omega_i}p_{ij}\notag\\
 &= \frac{1}{N}\sum_{i=1}^N\sum_{j\in\omega_i}\frac{e^{-d_{ij}^2}}{\sum_{k\neq i}e^{-d_{ij}^2}}\label{eq:nca-2},
\end{align}
 where $d_{ij}^2=\lVert A\mathbf{x}_i-A\mathbf{x}_j\lVert^2$.

By differentiating with respect to $A$, we obtain the gradient:
\begin{align}
  \frac{\partial f}{\partial A}=2A\sum_{i=1}^{N}\left(p_i\sum_{k=1}^Np_{ik}\mathbf{x}_{ik}\mathbf{x}_{ik}^{\textrm{T}} - \sum_{j\in\omega_i}p_{ij}\mathbf{x}_{ij}\mathbf{x}_{ij}^{\textrm{T}} \right)\label{eq:nca-grad},
\end{align}
where $\mathbf{x}_{ij} = \mathbf{x}_i - \mathbf{x}_j$.
\end{document}
