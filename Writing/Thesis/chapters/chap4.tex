\chapter{Reducing the computational cost}
\label{ch:reducing}

\begin{itemize}
	\item As mentioned in section , evaluating the objective function needs computing all the pairwise distances between the points. Also, the evaluating the gradient is expensive. This is done in $\mathcal{O}(N^2D^2)$ flops. So it is not trivial to successfully use NCA on large data sets.
	\item This chapter presents a wide palette of ideas and methods that can be applied to speed up the computations. Most of the methods rely on the fact that the learnt metric is low ranked.
	\item Every method presented can be regarded as an alteration of the original objective function. We basically change our objective function such that the new objective will have a reduced cost.
\end{itemize}

\section{Sub-sampling}
\label{sec:sub-sampling}

Sub-sampling is the simplest idea that can help speeding up the computations. For the training procedure we use a randomly selected sub-set $\mathcal{D}_n$ of the original data set $\mathcal{D}$:
 \[
 	\mathcal{D}_n = \{ \xB_{i_1},\cdots,\xB_{i_n} \} \subseteq \mathcal{D}.
 \]
 If $n$ is the size of the sub-set then the cost of the gradient is reduced to $\mathcal{O}(n^2D^2)$. After the projection matrix $\AB$ is learnt, it can be applied to the whole data set and all the data points are used for classification.

While easy to implement, this method discards a lot of information available. Also it is affected by the fact the sub-sampled data has a thinner distribution than the real data. 

\begin{figure}
  \centering
  \subfigure[Learnt projection $\AB$ on the sub-sampled data set $\mathcal{D}_n$.]{\label{fig:sub-sampling-1}\includegraphics[width=0.48\textwidth]{images/sub-sample-1}}
\subfigure[The projection $\AB$ applied to the whole data set $\mathcal{D}$.]{\label{fig:sub-sampling-2}\includegraphics[width=0.48\textwidth]{images/sub-sample-2}}
  \caption{Result of sub-sampling method on \texttt{wine}. There were used one third of the original data set for training, \textit{i.e.}, $n = N/3$. We note that the points that belong to the sub-set $\mathcal{D}_n$ are perfectly separated. But after applying the metric to the whole data there appear different misclassifications. The effects are even more acute if we use smaller sub-sets.}
  \label{fig:sub-sampling}
\end{figure}

\section{Mini-batches}
\label{sec:mini-batches}

The next obvious idea is to use sub-sets in an iterative manner, similar to the stochastic gradient descent method: split the data into mini-batches and train on them successively. Again the cost for one evaluation of the gradient will be $\mathcal{O}(n^2D^2)$ if the mini-batch consists of $n$ points.

\begin{algorithm} 
	\caption{Training algorithm using mini-batches formed by clustering} 
	\label{alg:mini-batches}  
	\begin{algorithmic} [1]                 % enter the algorithmic environment
		\REQUIRE Data set $\mathcal{D}=\{\xB_1,\cdots,\xB_N\}$ and initial linear transformation $\AB$.
		\REPEAT
			\STATE Project each data point using $\AB$:  $\mathcal{D}_\AB=\{\AB\xB_1,\cdots,\AB\xB_N\}$.
			\STATE Use either algorithm \ref{alg:fpc} or \ref{alg:rpc} on $\mathcal{D}_\AB$ to split $\mathcal{D}$  into $K$ mini-batches $\mathcal{M}_1,\cdots,\mathcal{M}_K$.
			\FORALL {$\mathcal{M}_i$}
				\STATE {Update parameter: $\AB\leftarrow \AB - \eta\frac{\partial f(\AB,\mathcal{M}_i)}{\partial\AB}$.}
				\STATE {Update learning rate $\eta$.}
			\ENDFOR
		\UNTIL {convergence.}
	\end{algorithmic}
\end{algorithm}

There are different possibilities for splitting the data-set:
\begin{enumerate}
	\item Random selection. In this case the points are assigned randomly to each mini-batch and after one pass through the whole data set another random allocation is done. As in section \ref{sec:sub-sampling}, this suffers from the thin distribution problem. In order to alleviate this and achieve convergence, large-sized mini-batches should be used (similar to Laurens van der Maaten's implementation). The algorithm is similar to Algorithm \ref{alg:mini-batches}, but lines 2 and 3 will be replaced with a simple random selection.
	
	\item Clustering. Constructing mini-batches by clustering ensures that the point density in each mini-batch is conserved. In order to maintain a low computational cost, we consider cheap clustering methods, \textit{e.g.}, farthest point clustering (FPC; \citealp{gonzalez1985}) and recursive projection clustering (RPC; \citealp{chalupka2011}). Algorithm 
	
	FPC gradually selects cluster centres until it reaches the desired number of clusters $K$. The point which is the farthest away from all the current centres is selected as new centre. The precise algorithm is presented below.
	
	\begin{algorithm} 
		\caption{Farthest point clustering} 
		\label{alg:fpc}  
		\begin{algorithmic}                    % enter the algorithmic environment
			\REQUIRE Data set $\mathcal{D}=\{\xB_1,\cdots,\xB_N\}$ and $K$ number of clusters.
			\STATE Randomly pick a point that will be the first centre $\cB_1$.
			\STATE Allocate all the points in the first cluster $\mathcal{M}_1 \leftarrow \mathcal{D}$.
			\FOR {$i=1$ to $K$}
				\STATE Select the $i$-th cluster centre $\cB_i$ as the point that is farthest away from any cluster centre $\cB_1,\cdots,\cB_{i-1}$.
				\STATE Move to the cluster $\mathcal{M}_i$ those points that are closer to its centre than to any other cluster centre: $\mathcal{M}_i = \left\{ \xB \in \mathcal{D} \;| \; d(\xB;\cB_i) < d(\xB;\cB_j), \forall j \neq i \right\}$
			\ENDFOR
		\end{algorithmic}
	\end{algorithm}
	
	This computational cost of this method is $\mathcal{O}(NK)$. However, we do not have any control on the number of points in each cluster, so we might end up with very unbalanced clusters. A very uneven split has a couple of obvious drawbacks: too large mini-batches will maintain high cost, while on too small clusters there is not too much too learn.
	
	So, as an alternative, RPC was especially developed to mitigate this problem. It constructs the clusters similarly to how the $k$-d trees are build (see section ). However instead of splitting the data set across axis aligned direction it chooses the splitting directions randomly (see algorithm \ref{alg:rpc}). So, because it uses the median value it will result in similar sized clusters and we can easily control the dimension of each cluster. The complexity of this algorithm is $\mathcal{O}()$.
	
	\begin{algorithm} 
		\caption{Recursive projection clustering} 
		\label{alg:rpc}  
		\begin{algorithmic}                    % enter the algorithmic environment
			\REQUIRE Data set $\mathcal{D}=\{\xB_1,\cdots,\xB_N\}$ and $n$ size of clusters.
			\IF {$N < n$}
				\STATE New cluster: $i\leftarrow i+1$.
				\RETURN current points as a cluster: $\mathcal{M}_i \leftarrow \mathcal{D}$.
			\ELSE
				\STATE {Randomly select two points $\xB_j$ and $\xB_k$ from $\mathcal{D}$.}
				\STATE {Project all data points onto the line defined by $\xB_j$ and $\xB_k$. (Give equation?)}
				\STATE {Select the median value $\tilde{\xB}$ from the projected points.}
				\STATE {Recurse on the data points above and below $\tilde{\xB}$: $\text{RPC}(\mathcal{D}_{>\tilde{\xB}})$ and $\text{RPC}(\mathcal{D}_{\le\tilde{\xB}})$.}
%							$\mathcal{D}_{>\tilde{\xB}}=\{\xB\in\mathcal{D}|\xB>\tilde{\xB}\}$
			\ENDIF
		\end{algorithmic}
	\end{algorithm}
	
	Note that we are re-clustering in the transformed space after one sweep through the whole data set. There are also other alternatives. For example, we could cluster in the original space periodically or we could cluster only once in the original space. However the proposed variant has the advantage of a good behaviour for a low rank projection matrix $\AB$. Not only that is cheaper, but the clusters resulted in low dimensions by using RPC are closer to the real clusters then applying the same method in a high dimensional space. 
	
	Further notes:
	\begin{itemize}
		\item The learning rate can be updated using various heuristics as presented in the section related to optimization.
		\item The convergence can be tested in various ways: stop when there is not enough momentum in the parameter space, when the function value doesn't vary too much or by using early stopping or a maximum number of iterations. Discuss all of these in practical issues section.
	\end{itemize}
	
\end{enumerate}

%\begin{center}
%	\begin{table}
%		\centering
%		\begin{tabular}{l}
%			\toprule
%			Training algorithm using mini-batches\\
%			\midrule
%			\textbf{Do}\\
%			Split data $\mathcal{D}$ into mini-batches: $\mathcal{M}_1,\cdots,\mathcal{M}_m$,\\
%			such that $\mathcal{M}_1\cup\cdots\cup\mathcal{M}_m=\mathcal{D}$\\
%			\textbf{For each} mini-batch $\mathcal{M}_i$\\
%			$\AB\leftarrow \AB - \eta\frac{\partial f(\AB,\mathcal{M}_i)}{\partial\AB}$\\
%			\textbf{End for}\\
%			\textbf{Until} we reach convergence\\
%			\bottomrule
%		\end{tabular}
%		\caption{Algorithm for training with mini-batches. The learning is done using gradeint ascent.}
%	\end{table}
%\end{center}


\section{Stochastic learning}
\label{sec:stochastic-learning}

The following technique is theoretically justified by stochastic approximation arguments. The main idea is to get an unbiased estimator of the gradient by looking only at a few points and how they relate to the entire data set.

More precisely, in the classical learning setting, we update our parameter $\AB$ after we have considered each point $\xB_i$ in the data set. In the stochastic learning procedure, we update $\AB$ more frequently by considering only $n$ randomly selected points. As in the previous case, we still need to compute $\{p_i\}_{i=1}^n$ using all the $N$ points. It should be further stressed the difference to the mini-batches approach; there, the contributions are calculated only between the $n$ points that belong to the mini-batch. 

The objective function that we need to optimize at each iteration and its gradient are given by the next equations:
\begin{align}
	f_\text{sNCA}(\AB) = \sum_{i=1}^n p_i\\
	\frac{\partial f_\text{sNCA}}{\partial \AB} = \sum_{i=1}^{n} \frac{\partial p_i}{\partial \AB}
	\label{eq:snca-grad}
\end{align}
This means that the theoretical cost of the stochastic learning method will scale with $nN$.

	\begin{algorithm} 
		\caption{Stochastic learning for NCA (sNCA)} 
		\label{alg:sNCA}  
		\begin{algorithmic}                    
			\REQUIRE Data set $\mathcal{D}=\{\xB_1,\cdots,\xB_N\}$, $n$ number of points to consider for the gradient estimation, $\AB$ initial linear transformation.
			\REPEAT
				\STATE Split data $\mathcal{D}$ into groups $\mathcal{M}_i$ of size $n$.\
				\FORALL {$\mathcal{M}_i$}
					\STATE {Update parameter using gradient given by Equation \ref{eq:snca-grad}: \STATE $\AB\leftarrow \AB - \eta\frac{\partial f_\text{sNCA}(\AB,\mathcal{M}_i)}{\partial\AB}$.}
					\STATE {Update learning rate $\eta$.}
				\ENDFOR
			\UNTIL {convergence.}
		\end{algorithmic}
	\end{algorithm}

Note: this idea might be used easily and robustly for on-line learning. Given a new point $\xB_{N+1}$ we update $\AB$ using the derivative $\frac{\partial p_{N+1}}{\partial \AB}$.

\section{Approximate computations}
\label{sec:approximate}

A straightforward way of speeding up the computations was previously mentioned in the original paper \citep{goldberger2004} and in some NCA related work \citep{weinberger2007, singh2010}. This method involves pruning small terms in the original objective function. We then use an approximated objective function and its corresponding gradient for the optimization process.

The motivation lies in the fact that the contributions $p_{ij}$ decay very quickly with distance:
 \[
 	p_{ij} \propto \exp\{-d(\AB\xB_i;\AB\xB_j)^2\}.
 \] 
 
 The evolution of the contributions during the training period is depicted in Figure.  We notice that most of the values $p_{ij}$ are insignificant compared to the largest contribution. This suggests that we might be able to preserve the accuracy of our estimations even if we discard a large part of the neighbours.

So, \citet{weinberger2007} choose to use only the top $m = 1000$ neighbours for each point $\xB_i$. Also they disregard those points that are farther away than $d_{\max}=34$ units from the query point: $p_{ij} = 0, \forall \xB_j$ such that $d(\AB\xB_i;\AB\xB_j)>d_{\max}$. While useful in practical situations, these suggestions lack of a principled description: how can we optimally choose $m$ and $d_{\max}$ in a general setting? We would also like to be able to estimate the error introduced by the approximations.

We correct those drawbacks by making use of the KDE formulation of NCA (see Section \ref{sec:cc-kde}) and adapting existing ideas for fast KDE \citep{deng1995,gray2003} to our particular application. We will use a class of accelerated methods that are based on data partitioning structures (\textit{e.g.}, $k$-d trees, ball trees). As we shall shortly see, these provide us with means to quickly find only the neighbours $\xB_j$ that give significant values $p_{ij}$ for any query point $\xB_i$. 
%Hence, we will be able to compute an approximated value of the true class-conditional probability $p(\xB_i|c) = \sum_{j\in c} k(\xB_i|\xB_j),\forall i,c.$
%
%A question still remains: given a point $\xB_i$ how can we \textit{quickly} ? It is clear that only the nearby points will contribute to $p_i$, while the points that are further away can be ignored. A framework that provides us with means of doing this is the $k$-d tree.

\subsection{$k$-d trees}
\label{subsec:k-d-trees}

The $k$ dimensional tree structure ($k$-d tree; \citealp{bentley1975}) organises the data in a binary tree using axis-aligned splitting planes. The $k$-d tree has the property to place close in the tree those points that live nearby in the original geometrical space. This makes such structures efficient mechanisms for nearest neighbour searches \citep{friedman1977} or range searches \citep{moore1991}.

There are different flavours of $k$-d trees. We choose for our application a variant of $k$-d tree that uses bounding boxes to describe the position of the points. Intuitively, we can imagine each node of the tree as a bounding hyper-rectangle in the $D$ dimensional space of our data. The root node will represent the whole data set and it can be viewed as an hyper-rectangle that contains all the data points (see Figure \ref{fig:kdtree-1}). Its children will contain disjoint subsets of the points contained by their parent (see Figure \ref{fig:kdtree-2}).

\begin{figure}
  \centering
  \subfigure[Root node]{\label{fig:kdtree-1}\includegraphics[width=0.45\textwidth]{images/kdtree-1}}
\subfigure[First level]{\label{fig:kdtree-2}\includegraphics[width=0.45\textwidth]{images/kdtree-2}}
\\
  \subfigure[Second level]{\label{fig:kdtree-3}\includegraphics[width=0.45\textwidth]{images/kdtree-3}}
\subfigure[Last level]{\label{fig:kdtree-4}\includegraphics[width=0.45\textwidth]{images/kdtree-4}}
  \caption{Illustration of the $k$-d tree with bounding boxes at different levels of depths.}
  \label{fig:kdtree}
\end{figure}

The tree building is done recursively and it starts from the root node. At each node we have to select which points of the current node are contained by each of the children; this is equivalent of saying we need to select the direction and the position of the split. Because the splitting directions are axis aligned, we can choose it from $D$ possible planes. One can choose this randomly or the use each of the directions in a successive manner. However, a more common approach, which we adopt is to split along the direction that presents the largest variance: 
\begin{align}
	\text{Splitting direction} \leftarrow \operatorname{argmax}_{d} (\max_ix_{id} - \min_ix_{id}).
	\label{eq:splitting-direction}
\end{align}
This results in a better clustering of the points. Otherwise it might happen that points situated in the same node can still be further away. We choose the splitting value to be the median value of the points contained by the current node. This guarantees us a balanced tree.

	\begin{algorithm} 
		\caption{$k$-d tree building algorithm} 
		\label{alg:kd-tree-build}  
		\begin{algorithmic}                    % enter the algorithmic environment
			\REQUIRE Data set $\mathcal{D}=\{\xB_1,\cdots,\xB_N\}$, $i$ position in tree and $m$ number of points in leaves.
			\IF {$N < m$}
				\STATE Mark node $i$ as leaf: \texttt{splitting\_direction(i)=-1}.
				\STATE Add points to leaf: \texttt{points(i)}$\leftarrow \mathcal{D}$.
				\RETURN
			\ENDIF
				\STATE {Choose splitting direction $d$ using Equation \ref{eq:splitting-direction}:}
				\STATE {\texttt{splitting\_direction(i)=}$d$.}
				\STATE {Find the median value $\tilde{x}_d$ on the selected splitting direction $d$:}
				\STATE {\texttt{splitting\_value(i)=}$\tilde{x}_d$.}
				\STATE {Determine points to the left $\mathcal{D}_{\le\tilde{\xB}}$ and to the right  $\mathcal{D}_{>\tilde{\xB}}$ of the median $\tilde{x}_d$.}
				\STATE {Continue recursive building on these subsets:}
				\STATE {Build left child \texttt{build\_kdtree(}$\mathcal{D}_{\le\tilde{\xB}}$\texttt{,2*i)}.}
				\STATE {Build right child \texttt{build\_kdtree(}$\mathcal{D}_{>\tilde{\xB}}$\texttt{,2*i+1)}.}
%							$\mathcal{D}_{>\tilde{\xB}}=\{\xB\in\mathcal{D}|\xB>\tilde{\xB}\}$
		\end{algorithmic}
	\end{algorithm}
	
	\begin{itemize}
		\item It is important to stress that the performance of $k$-d trees quickly degrades with the number of dimensions the data lives in. Also the structure of the data is important. Less structure means more searches. These aspects are illustrated in Figure. 
		\item So we will be able to efficiently use $k$-d trees only when learning a low-rank $\AB$ that projects the data points in a low dimensional space.
	\end{itemize}
	
%organize the data in a binary tree structure using axis-aligned splitting planes of the data space. Each node in the tree represents a data point and it also contains a splitting direction (this is usually denoted by an integer from 1 to $k$). The splitting plane is often chosen to be perpendicular on the dimension with the largest variance and to go through the median point (this results in a balanced tree). Furthermore, a non-leaf node has two successors: the sub-tree rooted at the left successor contains the points that are situated at the left of the splitting plane, while the sub-tree rooted at the right successor of the splitting plane contains the points that are situated at the right of the splitting plane. 

\subsection{Approximate kernel density estimation}
\label{subsec:approx-kde}

\begin{itemize}
	\item The following ideas are mostly inspired by previous work on fast kernel density estimators \citep{gray2003,shen2006}.
	\item The goal is to compute $p(\xB)$. In the kernel density estimation, we estimate the unknown probability density function as follows: $p(\xB_i) = \frac{1}{N}\sum_{j=1} Nk(\xB_i|\xB_j)$. If the number of samples $N$ is large then $p(\xB_i)$ can approximated to high degree of precision by discarding lots of the data. A question still remains: how can we do this in a sensible manner? 
	\item Given a query point $\xB$ and a group of points $G$ we can replace each individual contribution $k(\xB|\xB_j), \xB_j\in G$, with $k(\xB|\xB_g)$. This last quantity is specific for the group and will be common for all points in $G$. There are two ways to obtain a reasonable value for $k(\xB|\xB_g)$: either approximate $\xB_g$ first and then compute $k(\xB|\xB_g)$; here $\xB_g$ can be the mean of the points in the group. The second possibility is to approximate $k(\xB|\xB_g)$ directly; one possibility is $k(\xB|\xB_g) = \frac{\min_jk(\xB|\xB_j) + \max_jk(\xB|\xB_j)}{2}$. We prefer the second option, because it does not need to store mean of the points from $G$. Also, we will see that the minimum and maximum contributions have to be computed to decide whether to prune or not; so no further computational expense is introduced by using this approximation.
	\item We can see that the maximum error for each point in the group that is introduced by such an approximation is $\frac{\max_jk(\xB|\xB_j) - \min_jk(\xB|\xB_j)}{2}$. This can be controlled to be small; for example, we can decide to approximate only when the group is far away from the query point or when the kernel contribution is almost constant for the points within the group. However, it is better to consider the error relative to the total quantity we want to estimate. But we do not know the total sum before hand so we will use an upper bound $p(\xB) < p_{\text{SoFar}}(\xB) + \max k(\xB|\xB_j) $. This means that the order in which we accumulate it is important. 
	\item By using $k$-d trees our groups will be hyper-rectangles. So in order to prune we test that the largest and the smallest contribution within the hyper-rectangle varies a little. We start at with a large group, the root of the tree, that contains all the points and then recurs on its children until there is no need to and we can approximate. 
\end{itemize}

\subsection{Approximate KDE for NCA}
\label{subsec:approx-KDE-for-NCA}
	\begin{itemize}
		\item NCA was formulated as a class-conditional kernel density estimation problem. So we evaluate $p(\xB|c),\forall c$ and for each class we build a $k$-d tree. We will obtain an approximated version of the objective function. To obtain the gradient of this new objective function we can use Equation \ref{eq:nca-cc-kde-grad}. The derivative of $p(\AB\xB|c)$ will be different only for those groups where we do approximations. So, for such a group we obtain the following gradient:
		\begin{align}
			\frac{\partial}{\partial \AB} \sum_{j\in G} k(\AB\xB|\AB\xB_j) &\approx \frac{\partial}{\partial \AB} \frac{1}{2} \left\{\min_{j\in G} k(\AB\xB|\AB\xB_j) + \max_{j\in G} k(\AB\xB|\AB\xB_j)\right\}\notag\\
			& = \frac{1}{2}\left\{ \frac{\partial}{\partial \AB} k(\AB\xB|\AB\xB_c) + \frac{\partial}{\partial \AB} k(\AB\xB|\AB\xB_f) \right\},
		\end{align}
		where $\AB\xB_c$ denotes the closest point in $G$ to the query point $\AB\xB$ and $\AB\xB_f$ is the farthest point in $G$ to $\AB\xB$. Here we made use of the fact the kernel function is a monotonic function of the distance. So the closest point gives the maximum contribution, while the farthest point contributes the least.
	\end{itemize}
	
	
	\begin{algorithm} 
		\caption{Approximate NCA objective function and gradient computation} 
		\label{alg:cc-kde-nca}  
		\begin{algorithmic}                    % enter the algorithmic environment
			\REQUIRE Projection matrix $\AB$, data set $\mathcal{D}=\{\xB_1,\cdots,\xB_N\}$ and error~$\epsilon$.
			\FORALL {classes $c$}
				\STATE {Build $k$-d tree for the points in class $c$.}
			\ENDFOR 
			\FORALL {data points $\xB_i$}
				\FORALL {classes $c$}
					\STATE {Compute estimated probability $\hat{p}(\AB\xB_i|c)$ and the corresponding derivatives $\frac{\partial}{\partial \AB} \hat{p}(\AB\xB_i|c)$  using approximated KDE Algorithm:}
%					\STATE {\texttt{[p(c),dp(c)]=NCA\_recursive(kdtree(c))}}
				\ENDFOR
				\STATE Compute soft probability $\hat{p}_i \equiv \hat{p}(c|\AB\xB_i) = \frac{\hat{p}(\AB\xB_i|c_i)}{\sum_{c}\hat{p}(\AB\xB_i|c)}$.
				\STATE Compute gradient $\frac{\partial}{\partial \AB} \hat{p}$ using Equation \ref{eq:nca-cc-kde-grad}.
				\STATE Update function value and gradient value.
			\ENDFOR
		\end{algorithmic}
	\end{algorithm}
	
\section{Exact computations}
\label{sec:exact-computations}

The counterpart of the approximate methods are the exact computations. These are also obtained at some cost. We have to change our model such that it allows efficient exact methods. Again motivated by the rapid decaying squared exponential function, some of the contributions are almost zero. The idea is just to use a compact support kernel instead of the squared exponential kernel. After this is done, there will be a large number of points that will lie outside the support of the kernel and their contribution will be exactly zero instead of a very small value in the previous case.

The kernel needs to be differentiable, so we will use the simplest polynomial compact support kernel that satisfies this requirement. Of course, any other kernel that satisfies this two requirements can be used. This is given by the following expression:
\begin{align}
	k(u)=\begin{cases}
		\frac{15}{16}(a^2-u^2)^2& \mbox{if } u \in [-a;+a]\\
		0& \mbox{otherwise}.\\
	\end{cases}
	\label{eq:cs-1}
\end{align}
The expression from equation \ref{eq:cs-1} also integrates to one, so it can be used in the kernel density estimation context: $p(\xB_i|\xB_j) = k(d(\xB_i;\xB_j))$. 

For the simple replacement of the exponential kernel, we have preferred a simplified version of equation \ref{eq:cs-1}: 
	\begin{align}
		k(u)=\begin{cases}
				(1-u^2)^2& \mbox{if } u \in [-1;1]\\
				0& \mbox{otherwise}.\\
			\end{cases}
			\label{eq:cs-2}
	\end{align}
	
	The width of the kernel's support $a$ will be integrated in the projection matrix $\AB$, similarly as the widths for the Gaussian kernel are absorbed by $\AB$ for the classic NCA.

\begin{align}
	p_{ij} = \frac{k(d_{ij})}{\sum_{k\neq i} k(d_{ik})}.
\end{align}

Objective function will be: 
\begin{align}
	f_\text{CS}(\AB) = \sum_{i}\sum_{j\in c_i} p_{ij}
\end{align}

The gradient of the kernel:
\begin{align}
	\frac{\partial}{\partial\AB}k(d_{ij}) 
	&= 	\frac{\partial}{\partial\AB}\left[(1-d_{ij}^2)^2\cdot\mathrm{I}(\left|d_{ij}\right|\le 1)\right]\\
	&= -4\AB(1-d_{ij}^2)  \xB_{ij} \xB_{ij}^\mathrm{T} \cdot \mathrm{I}(\left|d_{ij}\right|\le 1),
\end{align}
where $d_{ij}^2 = d(\AB\xB_i;\AB\xB_j)$, $\xB_{ij}=\xB_i-\xB_j$ and $\mathrm{I}(\cdot)$ is the indicator function that return 1 when its argument is true and 0 when its argument is false.

The gradient of the new objective function:
\begin{align}
	\frac{\partial f_\text{CS}}{\partial \AB}=2\AB\sum_{i=1}^{N}
	\left(
	p_i \sum_{k=1}^N \frac{p_{ik}}{1-d_{ik}^2} \xB_{ik}\xB_{ik}^{\textrm{T}}
	- \sum_{j\in c_i} \frac{p_{ij}}{1-d_{ij}^2}\xB_{ij}\xB_{ij}^{\textrm{T}} 
	\right)
	\label{eq:nca-cs-grad},
\end{align}

The main concern with this method is what happens when points lie outside the compact support of any other point in the data set. So care must be taken at initialisation. One way would be to initialise with a very small scale $\AB$. However, this means that there is no gain in speed for at least the first iterations. It might be better to use the principal components for initialisation. 

\begin{itemize}
	\item Speedings comes from the fact only few gradient components need to be computed. A nice way of further accelerating this is again via $k$-d trees. This time they can be used for range search. 
	\item Note regarding the gradient: it was written in this form for simplicity. However, in an implementation it is not recommended to divide each term by $1-d_{ij}^2$; there might results cases that are not determined. Better is to compute the kernel values in two steps.
\end{itemize}

\section{NCA with compact support kernels and background distribution}
\label{sec:nca-cs-back}


\begin{figure}
  \centering\includegraphics[width=6cm]{images/nca-cs-back}
  \caption{Neighbourhood component analysis with compact support kernels and background distribution. Main assumption: each class is class is a mixture of compact support distributions $k(\xB|\xB_j)$ plus the normal background distribution $\mathcal{N}(\xB|\muB,\SigmaB)$.}
  \label{fig:cs-back}
\end{figure}

\begin{itemize}
	\item An extended variant of the previous scenario. We take care of the points that might remain unallocated by using a background distribution. This extension comes on naturally after the CC-KDE formulation. We can change the assumptions regarding the underlying distribution of a class $p(\xB_i|c)$. So, in this case, we consider that a point can be generated from a class by either a compact distribution from each point $k_\text{CS}(\xB_i|\xB_j)$ or by a global normal distribution that is specific to the whole class $\mathcal{N}(\xB_i|\muB_c,\SigmaB_c)$. So, the class-conditional distribution is given by the sum of these components:
	\begin{align}
		p(\xB_i|c) = \beta \mathcal{N}(\xB_i|\muB_c,\SigmaB_c) + (1-\beta) \frac{1}{N_c}\sum_{j\in c} k_\text{CS}(\xB_i|\xB_j),
		\label{eq:nca-cs-back-1}
	\end{align}
	where $\beta$ is the mixing coefficient between the background distribution and the compact support model and $\muB_c$ and $\SigmaB_c$ are the sample mean and covariance of the class $c$.
	\item It is unclear how we can choose the mixing proportion $\beta$. By setting $\beta = \frac{1}{N_c+1}$ we will give equal weights to the background distribution as to the compact-support distribution. Setting it too large means that the model will favour convex classes. On one hand, this might diminish NCA's power. One of its strengths lies in the ability to work on non-convex classes. On the other hand, some of the classes  This can be fitted as a parameter during the optimization process. The gradient with respect to $\beta$ can be easily derived and it is easy to evaluate it as the quantities required for this should also be computed for the function evaluation. 
	\item This model can be used by plugging equation \ref{eq:nca-cs-back-1} into the set of equations \ref{eq:nca-cc-kde-bayes}, \ref{eq:nca-cc-kde-obj} and \ref{eq:nca-cc-kde-grad} from section \ref{sec:cc-kde}. 
	\item We give here the gradient for the background distribution with respect to the point positions (the full derivation can be found in the appendix). 
	\item It might be useful to note that projecting the points $\{\xB_i\}_{i=1}^N$ into a new space $\{\AB\xB_i\}_{i=1}^N$ will change the sample mean $\muB_c$ to $\AB\muB_c$ and the sample covariance $\SigmaB_c$ to $\AB\SigmaB_c\AB^\mathrm{T}$. Hence, we have:
	\begin{align}
		\frac{\partial}{\partial \AB} \mathcal{N}(\AB\xB_i&|\AB\muB_c, \AB\SigmaB_c \AB^\mathrm{T}) = \mathcal{N}(\AB\xB_i|\AB\muB_c, \AB\SigmaB_c \AB^\mathrm{T})\notag\\
		&\times \{ -(\AB\SigmaB_c \AB^\mathrm{T})^{-1}\AB\SigmaB_c
		+\vB \vB ^ \mathrm{T} \AB \SigmaB_c - \vB (\xB - \muB_c)^\mathrm{T}
		\},
	\end{align}
	where $\vB = (\AB\SigmaB_c \AB^\mathrm{T})^{-1}\AB(\xB - \muB_c)$.
\end{itemize}
